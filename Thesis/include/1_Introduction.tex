\chapter{INTRODUZIONE}
I Social Media rappresentano oggi, il principale canale di comunicazione. Sono principalmente servizi utilizzati per creare e mantenere relazioni sociali con altre persone. Gli utenti delle piattaforme sociali sono cresciuti del 13\% nel 2020\footnote{https://www.primaonline.it/2021/01/29/319562/in-media-si-passano-online-7-ore-al-giorno-di-queste-meta-da-smartphone-nel-2020-13-di-utenti-sui-social-ecommerce-domina-il-beauty/}, con quasi mezzo miliardo di nuovi utenti. Il tempo speso sulle piattaforme social continua a crescere anche se in misura più contenuta rispetto agli ultimi anni, attestandosi a 2 ore e 25 minuti al giorno. In particolare, l'utilizzo di tali piattaforme è notevolmente aumentato durante la pandemia da COVID-19. Infatti, dall’inizio del lockdown è aumentato anche il tempo speso dagli utenti online, con una crescita sensibile dell’uso dei social networks e, in particolare, delle videochiamate. 

La rapida espansione dei social media ha incrementato anche le varie problematiche legate all'utilizzo di tali piattaforme, in particolare i problemi relativi alla privacy dei dati degli utenti, i quali in alcuni casi sono venduti a terze parti o vengono messi a rischio in seguito ad attacchi alle piattaforme. Tra gli altri problemi più importanti citiamo la presenza di fake news \cite{lazer2018science} (notizie senza alcun tipo di fondamento) o la possibilità di censura. Quest'ultimo è un aspetto particolarmente delicato poiché c'è un sottile confine tra la censura di contenuti offensivi/scorretti e la censura della libertà di espressione.

Le attuali Social Network, sono piattaforme centralizzate e i dati degli utenti vengono raccolti in server di proprietà della società che fornisce il servizio sociale.
Il problema della privacy è stato uno dei principali motivi che hanno portato alla definizione di soluzioni alternative, che potessero sfruttare architetture distribuite, quali reti P2P \cite{datta2010decentralized}, e in seguito la tecnologia blockchain \cite{guidi2020blockchain}.

Inizialmente sono nate le cosiddette Decentralized Online Social Media \cite{guidi2016didusonet}, ovvero Social Media decentralizzati grazie all'utilizzo di reti P2P \cite{datta2010decentralized}.
Successivamente, e grazie all'avvento della tecnologia Blockchain, le Decentralized Online Social Media sono evolute verso quelle che si definiscono oggi Blockchain Online Social Media \cite{guidi2020blockchain}.
Con il termine Blockchain Online Social Media (BOSMs) si fa riferimento alle attuali Social Networks che utilizzano la tecnologia blockchain per fornire servizi sociali e/o per incentivare gli utenti a produrre contenuti di valore. Sono piattaforme nuove dove la parte economica e la parte sociale sono integrate e si basano sui concetti di token e attention economy \cite{kazdin2012token} \cite{davenport2001attention}, secondo cui gli utenti che contribuiscono e che utilizzano la piattaforma vengono ricompensati con della criptovaluta.
Nelle BOSMs sono gli utenti a scegliere quali contenuti premiare per la loro qualità e sarà la piattaforma ad occuparsi di ricompensare l'autore del contenuto in questione ma anche coloro che hanno contribuito al successo di quanto pubblicato. E' facilmente intuibile come in questo sistema venga meno il concetto di servizio centralizzato, in quanto sono coloro che utilizzano il sistema a scegliere che cosa sia o meno di qualità. I vantaggi ottenuti dalle BOSMs sono, ad esempio, l'eliminazione della censura in quasi tutte le sue forme, la possibile verifica e tracciabilità delle transazioni effettuate e la possibilità di contrastare le fake news, che al giorno d'oggi sono sempre più comuni nell'ambito dei social.
E' importante far presente che l'applicazione della tecnologia Blockchain ai servizi social comporta anche degli svantaggi, alcuni legati alla natura decentralizzata della tecnologia, quali la scalabilità, altri legati alla difficoltà nella verifica dell'identità degli utenti.
Le blockchain ad oggi più note sono sicuramente Bitcoin ed Ethereum. Attualmente la tecnologia blockchain viene applicata a svariati campi, anche se notoriamente è utilizzata per pagamenti e trasferimenti di denaro. La blockchain può essere brevemente definita come un sistema che registra informazioni in una maniera tale da rendere difficile o impossibile modificare, hackerare o ingannare il sistema.

Negli ultimi 3 anni, il numero di BOSMs è incrementato notevolmente, così come le blockchain utilizzate. La BOSMs ad oggi più nota è Steemit \cite{guidi2020steem,guidi2021socioeconomic}. Steemit\footnote{\url{https://steemit.com/}}, con circa 1.5 milioni di utenti iscritti, è la Blockchain Social Media più utilizzata in questo momento ed è basata sulla blockchain Steem. Questa blockchain è progettata specificatamente per lo sviluppo e la gestione di un social media, ed è ottimizzata anche nelle prestazioni: ha infatti un'alta scalabilità e il suo sistema permette di ridurre l'intervallo di tempo di conferma di una transazione a pochi secondi.
Tra le altre possiamo menzionare Minds, BitClout, Hive Blog e PeakD.
La caratteristica comune di queste piattaforme è quello di fornire un ambiente alternativo alle classiche OSNs centralizzate. Purtroppo, il loro successo risulta ancora limitato se paragonato alle classiche OSNs. Questo potrebbe essere imputabile alla difficoltà di modificare le abitudini delle persone. Una novità in termini di BOSMs è rappresentato dalla piattaforma Yup, la quale si pone come obiettivo quello di valutare i contenuti che appaiono su piattaforme sociali, premiando economicamente i creatori dei contenuti migliori. L'idea alla base di Yup è innovativa nello scenario delle BOSMs. Infatti a differenza della maggior parte delle dApp social, come per esempio Steemit, non si propone come un'alternativa alle principali piattaforme sociali già esistenti (Facebook, Twitter, e così via), bensì come un'integrazione con queste ultime. L'utente può infatti utilizzare Yup semplicemente scaricando una estensione sul proprio browser che gli permette di iniziare a curare (valutare) contenuti e ricevere token ricompensa continuando ad utilizzare gli stessi social network che utilizzava precedentemente. E' fondamentale come, da un punto di vista anche solo psicologico, l'individuo non sia "costretto" ad abbandonare le sue piattaforme preferite. Naturalmente la distribuzione dei token ricompensa per i contenuti curati è anch'essa una conseguenza, forse la principale, del perché così tante persone utilizzino il servizio.

Lo scopo di questo tirocinio è quello di far luce sul funzionamento della dApp Yup, che risulta essere attualmente, una delle dApp sociali più utilizzate, con circa 6.000 utenti attivi al giorno\footnote{https://www.dapp.com/app/yup}. Ci siamo concentrati sulle possibili caratteristiche di un successo così importante, andando ad analizzare dettagliatamente il protocollo ed il meccanismo di ricompense, ed infine analizzando l'attività degli utenti. Nello specifico, abbiamo analizzato l'ambito sociale ed economico della piattaforma, quest'ultimo particolarmente importante nell'incentivare l'utilizzo della piattaforma stessa. Tra l'altro Yup si discosta proprio in questo ambito dalle numerose concorrenti anche solo per il valore del token che si riceve come ricompensa. Per esempio, durante \textit{Gennaio-Marzo 2021}, ha avuto un minimo di \textit{\euro1.28} ed un massimo di \textit{\euro5.85}. Se lo confrontiamo per esempio con i token utilizzati da social dApp concorrenti vediamo che: la moneta utilizzata su Hive, quindi da PeakD e Hive Blog, non ha mai superato nemmeno il valore di \textit{\euro1}, oppure quella utilizzata da Steemit, STEEM, solo recentemente, e per pochi giorni, ha superato il valore di \textit{\euro1}. Per conseguire l'obiettivo finale, abbiamo organizzato il lavoro di tesi in tre fasi distinte: un fase di studio necessaria alla comprensione della piattaforma, una fase di download dei dati, ed una fase di analisi. Durante le fase di studio del protocollo, abbiamo analizzato il meccanismo di ricompensa e gli Smart Contract utilizzati, focalizzandoci su quello principale (yupyupyupyup). Abbiamo inoltre valutato la presenza di 51 azioni differenti, che sono utilizzate dalla piattaforma per fornire i servizi sociali. Durante la fase di download, abbiamo collezionato le informazioni relative a tutte le azioni eseguite su Yup. Abbiamo poi recuperato le informazioni relative agli accounts registrati su Yup ed infine quelle relative ai singoli contributi sociali, ovvero i vari contenuti votati.
Nella fase di analisi ci siamo concentrati, come detto precedentemente, sull'analisi del protocollo, del meccanismo di reward, ed infine dell'attività degli utenti.
E' importante evidenziare le difficoltà individuate in questo studio. Una delle limitazioni maggiori incontrate durante lo sviluppo di questo lavoro di tesi è l'impossibilità di valutare il codice del protocollo, in quanto non pubblico, nonostante sia sviluppato su blockchain. Inoltre, la documentazione disponibile non è adeguata alla comprensione della totalità delle funzionalità. Infine, la piattaforma è nuova e in continuo sviluppo, di conseguenza alcune funzionalità e caratteristiche analizzate potrebbe subire modifiche così come è possibile che ne vengano aggiunte altre in futuro.

Questa tesi è organizzata come segue. Nel Capitolo \ref{chapter_background} viene presentato lo stato dell'arte nel campo delle Blockchain Online Social Networks ed una panoramica della tecnologia blockchain. Nel Capitolo \ref{platform_chapter} viene presentata la piattaforma Yup e lo scopo di questa tesi andando a descrivere le tre fasi di cui si compone. Nel Capitolo \ref{chapter_implementazione} vengono introdotti gli strumenti utilizzati per la fase di download della blockchain. Nel Capitolo \ref{analisi_marker} viene presentata l'analisi della piattaforma Yup. Il Capitolo 6 è dedicato ad un commento generale sul lavoro svolto e su possibili sviluppi con cui portarlo avanti.

Alla fine della relazione si trova l’Appendice dove sono riportate un insieme di azioni individuate nella fase di analisi della blockchain, e che non siamo riusciti a classificare per mancanza di documentazione.
%il codice del protocollo, non ci è possibile sapere con certezza il funzionamento di quest'ultimo così come la funzionalità di numerose azioni utilizzate al suo interno. Elencheremo tutte quelle individuate ma descriveremo esclusivamente quelle di cui è stato possibile appurare lo scopo grazie ai test sopracitati.
%\\
%\\
%La scelta di effettuare un caso di studio su Yup è conseguenza dell'attività presente su tale piattaforma. La dApp in questione ricopre infatti posizioni notevoli sia se la paragoniamo a sue dirette concorrenti (social dApp), sia se la paragoniamo in generale ad altre dApp che utilizzano o meno la stessa blockchain.
%\\
%\\
%Secondo DappRadar.com:

%\begin{itemize}
%    \item \textbf{\#1 social dApp}
%    \item \textbf{\#2 dApp su EOS}
%    \item \textbf{\#19 dApp in generale}
%\end{itemize}

%Questa classificazione prende in considerazione il numero di utenti, volendo essere più specifici il numero di differenti indirizzi wallet, che interagiscono con lo smart contract. Oltretutto si discosta notevolmente, sempre secondo tale parametro, dalle immediate rivali in ambito social. Per esempio, \textbf{Steemit}, in \#2 posizione, ha solamente $4.444$ utenti oppure \textbf{PeakD}, in \#3 posizione, ne ha addirittura solamente $1.848$, contro i $8.828$ di Yup.
%\\
%\\

