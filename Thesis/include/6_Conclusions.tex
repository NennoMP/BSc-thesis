\chapter{CONCLUSIONI E LAVORI FUTURI}
Nel mondo reale, i social networks e i social media giocano un ruolo fondamentale all'interno delle nostre vite. Il connubio tra social media e blockchain costituisce un’alternativa interessante, che, col tempo, potrebbe sostituire i meccanismi centralizzati tipici dei servizi Internet, ai quali si è più abituati, con meccaniche di funzionamento decentralizzate.
Lo scopo di questo tirocinio era quello di far luce sul funzionamento della dApp Yup, che risulta essere attualmente, una delle dApp sociali più utilizzate, con migliaia di utenti attivi al giorno\footnote{https://www.dapp.com/app/yup}. Ci siamo concentrati sull'analisi del protocollo, sul meccanismo di reward, ed infine sull'attività degli utenti. 
In ambito "social" è stata effettuata un'analisi della popolarità della piattaforma, evidenziando le azioni di creazione account, gli utenti attivi e la loro attività, la distribuzione dei voti per piattaforma e i profili più rilevanti su Yup. Spostandoci su un ambito più "economico" del protocollo ci siamo concentrati sull'utilizzo del servizio di Bridge e dei token posseduti e riscossi. Insieme costituiscono un'osservazione esaustiva dei profitti che gli utenti hanno tratto dall'utilizzo di Yup, sono infatti inclusi sia coloro che al momento mantengono i propri token su EOS, sia coloro che invece li hanno trasferiti su Ethereum.

Queste analisi hanno evidenziato delle particolarità interessanti del protocollo e dell'utilizzo che gli utenti fanno della piattaforma Yup. Tenendo conto del fatto che il codice dello Smart Contract non è pubblico e dell'assenza di documentazione relativa alle azioni che lo regolano, siamo riusciti, tramite l'analisi dei dati a nostra disposizione di capire la funzione di alcune azioni e a capire l'utilizzo tipico che ne fanno gli utenti.
In particolare, le nostre analisi hanno evidenziato l'esistenza di molti account, chiamati Mirror, non gestiti da umani ma che ripetono le azioni che alcune celebrità fanno sui propri account social.
Abbiamo inoltre scoperto che tra i contenuti più popolari spiccano quelli legati al mondo blockchain e investimenti in criptovalute e che gli stessi utenti Yup sfruttano un meccanismo, chiamato Bridge, per trasferire i propri fondi su Ethereum per poi utilizzarli come meglio credono.
Lo studio dell'attività degli utenti ha mostrato che in alcuni momenti lo smart contract di Yup su EOS non è stato in grado di sostenere il carico di lavoro, al punto che molti utenti si sono lamentati sui canali ufficiali che le ricompense per le loro attività non arrivano.

Abbiamo scaricato i dati relativi a Yup interfacciandosi con la blockchain EOS. Il dataset utilizzato per la fase di analisi prende in considerazione tutte le azioni presenti sullo smart contract principale (\textbf{yupyupyupyup}) dal \textit{15 Settembre 2018}, periodo di inizio della sua attività, fino al \textit{28 Febbraio 2021}. Abbiamo recuperato un numero di azioni pari a $23.567.430$. 
I dati degli utenti Yup sono stati reperiti in due modalità: conteggiando il numero di \textbf{createacct} sullo smart contract e scaricando il JSON contente tutti gli utenti utilizzando le API ufficiali. Abbiamo valutato come il numero di account presenti nel JSON è notevolmente superiore al numero di createacct, ciò significa che esistono account non generati tramite l'interazione con il protocollo. Abbiamo definito questi utenti Mirror e nelle analisi abbiamo valutato i risultati includendo ed escludendo tali account Mirror. In dettaglio, abbiamo appurato l'esistenza di \textit{16.484} account Yup, ma solo per \textit{11.045} di questi è stato possibile ritrovare una corrispondente azione di creazione account nel protocollo.



%\todo[inline]{Tutta questa introduzione deve essere estesa, andando ad integrare qui cosa si è fatto con l'analisi. Come un riassunto di tutto.}

%In ambito "social" è stata effettuata un'analisi della popolarità della piattaforma, evidenziando le azioni di creazione account e gli utenti attivi, e dell'attività degli utenti, evidenziando la distribuzione dei voti per piattaforma e profili a questa relativi. Spostandoci su un ambito più "economico" del protocollo ci siamo concentrati sull'utilizzo del servizio di Bridge e dei token posseduti e riscossi. Insieme costituiscono un'osservazione esaustiva dei profitti che gli utenti hanno tratto dall'utilizzo di Yup, sono infatti inclusi sia coloro che al momento mantengono i propri token su EOS, sia coloro che invece li hanno trasferiti su Ethereum.
%Nella maggior parte dei casi abbiamo sempre differenziato le analisi includendo e poi escludendo gli account Mirror, in modo da avere nel primo caso l'attività complessiva e nel secondo quella reale, visto che si considerano solo gli utenti effettivi.

%Durante la fase di analisi ci siamo concentrati principalmente nell'analisi economica e sociale della piattaforma, andando a valutare l'attività degli utenti. In dettaglio, abbiamo osservato vari aspetti del protocollo, tra cui l'utilizzo delle varie azioni, l'attività in termini di utenza, le varie piattaforme sociali coinvolte, ecc. Nel Capitolo \ref{analisi_marker}, sono riportate tutte le analisi effettuate e le considerazioni sui risultati ottenuti.
%\\
%\\
Considerando solo il periodo successivo alla conclusione della fase di Beta (\textit{6 Ottobre 2020 - Febbraio 2021}), risulta una media di circa \textit{700} utenti mensili socialmente attivi, ovvero che hanno effettuato almeno un'azione in ambito sociale in quel mese. Di questi è stato osservato un picco particolarmente alto nel mese di Settembre, specificamente tra il 23 e 24 del mese. Queste ultime considerazioni sono relative ai soli account non-Mirror, tuttavia nel capitolo dedicato alla fase di analisi tratteremo anche quelle che tengono in considerazione gli account di sistema.

Restando in ambito sociale abbiamo individuato contenuti votati appartenenti a 1.884 piattaforme diverse. Tra queste dominano tre delle quattro piattaforme integrate (in ordine Twitter, Youtube e Reddit), anche se è possibile notare una notevole presenza di quelle dedicate al commercio di NFT (come Rarible e SuperRare). Tra le piattaforme social, molti utenti mostrano grande interesse per i profili o gruppi che trattano argomenti legati al mondo cripto (Elon Musk e Binance su Twitter, fanatici ed esperti di criptomonete su Youtube). Caso particolare quello di Reddit, in cui troviamo, tra i più popolari, anche subreddit con materiale NSFW.

Spostandoci infine sull'ambito economico abbiamo analizzato il funzionamento del meccanismo di ricompense e la distribuzione dei token tra i vari utenti. Nel primo caso è stato notato come quelle destinate ai Creatori non vengano distribuite agli account Mirror corrispondenti bensì accumulate sull'account \textbf{yupcreators1}, apparentemente per una futura distribuzione quando più Creatori riscatteranno tali account o si iscriveranno alla piattaforma. Nel secondo è stato possibile osservare come gli utenti che hanno riscosso o possiedono più token siano, nella maggior parte dei casi, utenti che, oltre ad utilizzare il servizio, forniscono liquidità su Uniswap.
Abbiamo potuto inoltre constatare che i 3 utenti più ricchi della piattaforma posseggano oltre il 50\% dei token e che questo trend si sta estremizzando nel tempo.


\section{Lavori futuri}
Senza dubbio ci sono ancora molti aspetti della dApp Yup che richiedono un'analisi approfondita e che purtroppo non è stato possibile trattare in questa tesi. Di seguito elenchiamo alcuni possibili lavori futuri:

\begin{itemize}
    \item \textbf{Analizzare l'utilizzo delle risorse del protocollo (RAM, CPU, NET):} sarebbe interessante vedere il suo funzionamento in relazione alle risorse disponibili, visto che in alcuni casi è stato osservato un temporaneo blocco del servizio per via di una saturazione della RAM o CPU che ha causato probabilmente dei cali improvvisi di utilizzo della piattaforma, e nello specifico anche gli spostamenti di RAM, visto che può essere comprata e venduta.
    \item \textbf{Indagine su eventuale presenza di bot:} l'utilizzo di bot per effettuare dei voti è altamente probabile visto il grande incentivo economico che si ha dall'essere i primi a votare un contenuto e che in alcuni casi è facile prevedere quali contenuti saranno maggiormente votati. Questa analisi potrebbe essere condotta effettuando un controllo sul timestamp dell'azione nell'explorer, essendo in tempo reale, e quello del contenuto valutato utilizzando tecniche di analisi di time series.
    \item \textbf{Indagine account sospesi:} di tanto in tanto alcuni utenti vengono sospesi per aver violato determinate regole. A causa della sospensione questi utenti non possono esprimere voti e la loro Influenza viete settata a 0. Sarebbe interessante provare a capire quali siano le motivazioni che portano a questi ban temporanei e probabilmente sarebbe un buon inizio anche per l'analisi sui bot citata sopra.
    \item \textbf{Analisi sull'utilizzo dei token migrati su Ethereum:} pagando il Gas di Ethereum, e di conseguenza avendo accesso ai dati della blockchain, dovrebbe essere possibile rintracciare i token Yup che vengono trasferiti con il Bridge. Si potrebbe quindi scoprire come gli utenti spendono i token migrati su Ethereum, per esempio utilizzati per il mint di un NFT, scambiati per altre criptovalute su un exchange e così via.
    \item \textbf{Analisi transfer utente-utente:} essendo possibile inviare token ad un altro utente EOS senza commissioni, potrebbe essere interessante vedere, magari tramite l'utilizzo della teoria dei grafi, se ci sono utenti che periodicamente effettuano degli scambi tra loro. Un'indagine di questo tipo permetterebbe l'avanzamento di ipotesi sulla presenza di multi-account (una singola persona che gestisce più account Yup) o sull'utilizzo di tale funzionalità come pagamento tra individui.
\end{itemize}